\subsubsection{Generación de casos de prueba}

Los estudiantes deberán implementar un generador de casos de prueba en Python llamado \texttt{testcases\_generator.py}, ubicado en:

\begin{center}
    \texttt{code/implementation/scripts/testcases\_generator.py}
\end{center}

\textbf{Requisitos del generador:}
\begin{itemize}
    \item Debe generar archivos de entrada válidos para el problema de productividad de desarrolladores.
    \item Cada archivo de entrada debe llamarse:
    \[
        \texttt{testcases\_\{n\}\_\{i\}.txt}
    \]
    donde:
    \begin{itemize}
        \item \texttt{\{n\}} representa la cantidad de empleados del caso de prueba.
        \item \texttt{\{i\}} es un identificador único para diferenciar múltiples casos con la misma cantidad de empleados.
    \end{itemize}
    \item Formato del archivo de entrada:
    \begin{itemize}
        \item La primera línea contiene un entero $n$, la cantidad de empleados, con $1 \le n \le 10^4$.
        \item Las siguientes $n$ líneas contienen tres elementos por empleado:
        \[
        A_i \ B_i \ C_i
        \]
        donde:
        \begin{itemize}
            \item $A_i$ es la productividad del empleado $i$ cuando programa con su lenguaje favorito (entero entre $-10^9$ y $10^9$).
            \item $B_i$ es la productividad del empleado $i$ cuando programa con un lenguaje distinto al favorito (entero entre $-10^9$ y $10^9$).
            \item $C_i$ es el lenguaje favorito del empleado $i$, representado por una letra minúscula del alfabeto inglés (\texttt{a--z}), sin espacios.
        \end{itemize}
    \end{itemize}
    \item El generador debe permitir crear múltiples casos de prueba de manera automática, variando la cantidad de empleados, los valores de productividad y los lenguajes favoritos.
\end{itemize}

\begin{mdframed}
    Los archivos generados serán utilizados por los algoritmos implementados en C++ para probar la corrección y eficiencia de las soluciones de fuerza bruta, greedy y programación dinámica.
\end{mdframed}
